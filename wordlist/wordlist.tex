\documentclass[a4paper, 12pt]{article}
\setlength{\parindent}{0em}

\usepackage{amssymb}
\usepackage{amsmath}
\usepackage[document]{ragged2e}
\usepackage{comment}
\usepackage{upgreek}
\usepackage{gb4e}
\noautomath
\usepackage{multicol}
\usepackage{tipa}
\usepackage{qtree}
\usepackage{tikz-qtree}
\usepackage{textcomp}
\usepackage{graphicx}
\usepackage{hyperref}
\usepackage{pifont}
\usepackage[mathscr]{eucal}
\usepackage[margin=.75in]{geometry}
\newcommand{\xmark}{\ding{55}}%5
%\pagenumbering{gobble}

\def\HS{\space\space}

\begin{document}

Ryan Sie\\
LIGN 110\\
Research Paper Wordlist draft\\


\section{Consonants}

(21 Consonant phonemes)

\begin{center}
	\begin{tabular}{|p{2.15cm}|*{8}{c |}}
		\hline
		& Bilabial & Labio-dental & Dental & Alveolar & Post-Alveolar & Palatal & Velar & Glottal \\ \hline
		Stop & p\HS b &  & \textsubbridge{t} &  d &  & & k g & (\textipa{P})  \\ \hline
		Affricate & & & & & \textteshlig \HS \textdyoghlig & & & \\ \hline
		Fricative & & f & & s z & \textipa{S} & & & h \\ \hline
		Nasal & m & & & n & & \textltailn & \textipa{N} & \\ \hline
		Flap/Trill & & & & r & & & & \\ \hline
		Approximant & w & & & & & j & & \\ \hline
		Lateral Approximant & & & & l & & & & \\ \hline	
	\end{tabular}	
\end{center}

length limit ~ (21 + 6) * 1.25 = 34 words \\ \medskip

Below are groups of near-minimal pairs to demonstrate separate phonemes

\bigskip 

p \qquad \textipa{["pal{\textsci}N]} - paling - 'most'

b \qquad \textipa{["bali]} - Bali - (an island)

m \qquad \textipa{["malu]} - malu - 'ashamed'

w \qquad \textipa{["wali]} - wali - 'guardian'

f \qquad \textipa{["fak\|[ta]} - fakta - 'fact'

\medskip


z \qquad \textipa{["za\|[t\textcorner]} - zat - 'substance'

s \qquad /\textipa{sabuk}/ \qquad \textipa{[sa"b{\textupsilon}P]} - sabuk - 'belt, sash'

\textesh \qquad  \textipa{["Saban]} - syaban - 'Sha'ban, 8th month of the Islamic calendar'

r \qquad \textipa{[ra"bu]} - rabu - 'Wednesday'

l \qquad \textipa{["labu]} - labu - 'squash, gourd'

\medskip

\textteshlig \qquad \textipa{["{\textteshlig}ari]} - cari - 'find'

\textdyoghlig \qquad \textipa{["{\textdyoghlig}ari]} - jari - 'finger'

\textipa{\|[t} \qquad \textipa{["\|[tari]} - tari - 'dance'

d \qquad \textipa{["dari]} - dari - 'from'

\medskip

n \qquad \textipa{[na"rasi]} - narasi - 'narrative'
 
\textltailn \qquad /\textipa{{\textltailn}aris}/ \qquad \textipa{["{\textltailn}ar{\textsci}s]} - nyaris - 'almost'

\textipa{N} \qquad /\textipa{NantuP}/ \qquad \textipa{[Nan"\|[t{\textupsilon}P]} - ngantuk - 'sleepy'

\medskip

j \qquad \textipa{["jaN]} - yang - 'that, which'

k \qquad /\textipa{kantoN}/ \qquad \textipa{["kant{\textopeno}N]} - kantong - 'bag'

g \qquad /\textipa{gantuN}/ \qquad \textipa{["gantoN]} - gantung - 'hanging'


\medskip

\textglotstop \qquad \textipa{["maPaf]} - ma'af - 'sorry'

h \qquad \textipa{["paha]} - paha - 'thigh'

\medskip
\section{Vowels}

(6 Vowel phonemes) \\ \medskip

\begin{center}
	\begin{tabular}{|p{2cm}|*{3}{c |}}
		\hline
		& Front & Central & Back \\ \hline
		 High & i & & u \\ \hline
	     High-  & (\textsci) & & (\textupsilon) \\ 
	     mid  & e &  & o\\ \hline
		 Mid & & \textschwa & \\ \hline
		 Low-mid  & (\textepsilon) & & (\textopeno) \\ \hline
		Low &  & a & \\ \hline
	\end{tabular}	\\
\end{center}

Below are groups of near-minimal pairs to demonstrate separate phonemes

\medskip


i \qquad \textipa{["biru]} - biru - 'blue'

a \qquad \textipa{["baru]} - baru - 'new'

u \qquad \textipa{[bu"ruP]} - buruk - 'bad, poor'

o \qquad \textipa{[""boro"budur]} - Borobudur - 'name of a temple in Central Java'

\medskip

e \qquad \textipa{["bebas]} - bebas - 'free' 

\textschwa \qquad \textipa{[b@"brapa]} - beberapa - 'several'

\section{Allophones}

\subsection{Consonants}

- All consonant allophones are also independent phonemes. (Mention some examples)

/k/ - [k] and [\textglotstop] \\
 - [\textglotstop] \HS appears syllable finally in the ultimate syllable of a word as an allophone of /k/

\bigskip

\subsection{Vowels} 

The 4 additional vowels ([\textsci, \textepsilon, \textopeno, \textupsilon]) in parentheses sometimes appear as allophones, typically being lowered (from [i, e, o, u]) in a final closed syllable, or a in a penultimate syllable which is followed by a final closed syllable whose vowel agrees in height (Soderberg, 2008). 

\medskip


/e/ - [e] and [\textepsilon] \\ \medskip

/sepele/ - \textipa{[s@"pele]} - sepele - 'not important'

/oleh/ - \textipa{[o"lEh]} - oleh - 'by'

\medskip


/o/ - [o] and [\textopeno] \\ \medskip

/toko/ - \textipa{["\|[toko]} - toko - 'store'

/tokoh/ - \textipa{[\|[tO"kOh]} - tokoh - 'figure, character'

\medskip


\subsection{Stress}

/pintu/ - \textipa{["pin\|[tu]} - pintu - 'door'

/p{\textschwa}rut/ - \textipa{[p@"ro\|[t]} - perut - 'stomach

/s{\textschwa}b{\textschwa}lum/ - \textipa{[""s@b@"lom]} - sebelum - 'before'

\section{Sentence}

Tolong buku yang biru dibungkus.

\medskip

\textipa{["tOlON "buku "jaN "biru di"buNk{\textupsilon}s]}

\medskip

"Please wrap the blue book."

\end{document}